\documentclass[11pt]{article}
\usepackage[utf8]{inputenc}


\title{Informática en el tiempo}
\author{Juan David Londoño Castaño\\% Name author
    \href{\textbf{}{juan.londono49@udea.edu.co}}
    }
\date{March 2020}
    
\begin{document}


\maketitle

\section{Introduction}

Si bien sabemos, el mundo que vivimos al día de hoy ha permitido tener un acercamiento más común al entorno físico en el cual se sumerge la existencia humana, pero son esos pasos agigantados por medio del término “informática” que engloba toda la evolución actual y no deja de ser el termino más común que acompaña la trayectoria de la existencia en cuanto se trata a desarrollo.

Devolvamos en el tiempo y exploremos que fue el causante de tal impacto actualmente.

Entre el siglo 18 e inicios del siglo 19, el mundo presentaba una cantidad increíble de búsqueda por llegar a la solución de problemas que iban asomando en el mundo de las matemáticas y a medida que estos pensadores mostraban nuevas posturas en el campo de las matemáticas, el querer facilitar los procesos de solución, un nuevo campo iba tomando postura.

Tras la idea de crear nuevas formas de solucionar problemas que presentaba el mundo de las matemáticas, surgió la necesidad de crear sistemas capaces de solucionar aquello que solo se encontraba en el papel mediante la invención de maquinas por medio de sistemas de engranajes, perforaciones en tarjetas y serian estos el rol mas importante que hizo tomar el inicio de todo aquello que hoy se conoce como informática.
Allí donde ya el rumbo de las matemáticas no solo era la idea de querer alcanzar la perfección, entender el entorno y querer llegar al más allá de lo que se es conocido, se debía tener en cuenta que ya el mundo se estaba apropiando de esta ciencia y la implementación de todo aquello dicho por pensadores, se iba a topar con mundo que pone aprueba todo lo existente.

Fue la implementación de todo lo ya mencionado, que hizo que fuese necesario la creación de la informática, esa idea de juntar información con procesos automáticos, permitió que maquinas como la calculadora, las computadoras, las máquinas de escribir y muchas mas que conoces al día de hoy, pudiesen ser comprendidas.

La informática datada desde sus inicios como simple procesamiento de la información por medio de máquinas, que  hizo que pensadores como Alan Turing (padre de la informática), tomaran de manera más común la idea de comunicarnos por medio de este, tras basarse por esta necesidad fue necesario crear nuevas formas que permitan el entendimiento de este sistema y fue allí donde surgió la idea de crear un lenguaje, pensar como máquina, limitar las decisiones y salirnos de todo lo abstracto que es el mundo de los axiomas.
La creación de un sistema capaz de interactuar con máquinas, hizo que todo lo matemático dicho hasta el día de su creación “informática”, fuese implementado de manera muy puntual, ya que a medida que se adentraban en el mundo la informática contextos como el infinito no tuviese cavidad pues los sistemas informáticos son procesos limitado y finitos (enunciado por Alan Turing).
Este simple hecho que a lo largo desencadeno todo un mar de posibilidades para el desarrollo humano, permitió que este campo fuera el puente para la comunicación de lo físico, pues el mero hecho de tomar una temperatura por medio de dispositivos electrónico ya esta aplicando informática. 

Realizar algo automáticamente es algo que hoy día vemos muy común, las maquinas que empacan, que dispensan líquidos, incluso maquinas que permiten la interacción como si fuesen humanos traen de por si alguien de tras que hizo posible esto y se es conocido como PROGAMADOR, quien permite la clara implementación de un lenguaje a un dispositivo mecánico que ejecutara las tareas que le fueron asignadas.

Al abarcar un nuevo mundo sumergido por la informática, el hombre se tomó la idea de crear nuevos dispositivos que en conjunto fuesen capaces de procesar información y una mejor eficiencia a la hora de realizar trabajo surgiendo así los tubos al vacío, que ya no serían la implementación de lo mecánico sino la asociación de dos cualidades físicas de nuestro entorno como, la mecánica y la electricidad. Y es allí un punto de partida para la creación de nuevas disciplinas como la electrónica.

Siempre conservando a la idea todo lo matemático, la informática ha permitido el surgimiento de nuevas disciplinas pues la necesidad de seguir explorando y hacer llegar esto a todos, no ha limitado la idea de crear nuevos dispositivos como el transistor, es el claro ejemplo de querer manejar el medio físico como información y ha permitido que lleguemos a lugares que solo empezó como la idea de unos cuantos, desde tener celular hasta tomar fotos de estrellas a millones de años luz de nosotros.

No solo ha desencadenado pequeños dispositivos pues los que hoy se conocen son demasiados, sino mas bien que hizo posible saltar la barrera de limitación que se tenia en años atrás y que la búsqueda por conocer todo aquellos que nos rodea va a ser imparable.

Ahora vemos que las matemáticas no solo son un montón de numero amontonados que simple mente designa algo, vemos que, ha sido el hermano fiel para la implementacion de la informatica, pues el ella siempre se esta plasmando todo el conjunto de ideas matematica que en tiempo atras solo era un dilema existencial, y que todo aquello que esta lejos de alcanzar, la informatica con un simple bucle nos pone mas cerca.

 
 


\section{Conclusión}

Vemos que básicamente el creciente actual y lograr que el medio sea más simple de entender, lo ha dado la informática, pues imagínate un mundo en donde la información no fue trabajada más rápidamente, pues fue esa idea que surgió en medio de tantos choques y aportadores matemáticos que querían saber el todo, ahora soy yo quien quiero entender el mundo desde mi casa al lado de un laptop programando y con tan solo un click.

\section{Referencias}
\href{https://www.youtube.com/watch?v=gbMZspfWTXI}\\
\href{http://gaceta.rsme.es/abrir.php?id=97}\\
\href{https://blogs.elpais.com/turing/2012/07/turing-el-nacimiento-del-hombre-1912-la-maquina-1936-y-el-test-1950.html}\\
\href{https://www.bbc.com/mundo/noticias-45300219}\\
\href{http://bibliotecadigital.ilce.edu.mx/sites/ciencia/volumen3/ciencia3/112/htm/sec_22.htm}\\


\end{document}
